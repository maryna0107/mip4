% Metódy inžinierskej práce

\documentclass[10pt,twoside,slovak,a4paper]{article}

\usepackage[slovak]{babel}
%\usepackage[T1]{fontenc}
\usepackage[IL2]{fontenc} % lepšia sadzba písmena Ľ než v T1
\usepackage[utf8]{inputenc}
\usepackage{graphicx}
\usepackage{url} % príkaz \url na formátovanie URL
\usepackage{hyperref} % odkazy v texte budú aktívne (pri niektorých triedach dokumentov spôsobuje posun textu)

\usepackage{cite}
%\usepackage{times}

\pagestyle{headings}

\title{Multiplayer games: the opportunity to develop your social skills?\thanks{Semester project in the subject Methods of engineering work, the academic year 2022/23, leading: Vladimír Mlynarovič}} 

\author{Maryna Kolesnykova\\[2pt]
	{\small Slovenská technická univerzita v Bratislave}\\
	{\small Faculty of informatics and information technologies}\\
	{\small \texttt{xkolesnykova@stuba.sk}}
	}

\date{\small 6. December 2022} 



\begin{document}

\maketitle

\begin{abstract}

The article explores how multiplayer games influence people’s behavior and abilities to communicate with each other. Recent research on human interaction wouldn’t be completely clear without examining the field of online entertainment (computer gaming in particular). A brief overview of the psychologist’s conclusion and the professional player’s interview will form the base for the analysis. Another crucial point of this work is the target audience The focus then shifts back to team-player game development, highlighting the last few years of explosive growth and its impact on modern society. The difference between face-to-face communication and text or voice massaging is discussed by comparing brain activity while completing the required game tasks and normal physical ball games.
\end{abstract}



\section{Introduction}

  With the rapid progress in the field of information technology and the popularization of new and diverse devices (especially those used in the field of entertainment), it is difficult not to note the close intertwining of virtual reality and the real world. Communication using social networks has become an integral part of our lives. The growing demand for online entertainment (especially games) has forced developers, programmers, and other people, closely related to this field, to take a decisive step forward and spread the means of communication in online games. The most difficult part of the plan remained the implementation of convenient means of communication, which would be practical to use during the execution of the tasks. While the rest of the world is slowly gaining access to the technologies, traditions, and interactions of digital society, corporate organizations are striving to establish themselves as the main owners of cyberspace. The desire of corporations to dominate the Internet is motivated by the very nature of digitalization, which reinforces a social process in which the production and distribution of information become the most important economic activity of society, where information technology begins to function as a key infrastructure for all industrial production and service provision, and in which information itself becomes a product, which is sold worldwide. Comparing the advantages and disadvantages of individual areas of cyberspace development helps to understand how these areas of human activity and artificial intelligence affect modern society. The spread and popularization of eSports require constant improvement of communication tools, affecting home gamers' entertainment needs. The improvement of small spheres of cyberspace is reflected in the entire Internet community, especially children and teenagers, who, according to research, are the most active users of the Internet space.

  In the era of active development of information technologies and the creation of one's own "virtual self", communication in virtual reality and the development of communication skills in this environment is problem number one in terms of importance and concern among the population of European countries.  Texting is the main way of communicating in online gaming structures when voice and video chats have just started to populate mainstream gaming platforms. The ability to non-verbally perceive information, focusing on controlling simulations and other types of games, creates the basis for the development and implementation by entrepreneurs of a platform that provides guaranteed income in the future and acts as a trigger for the improvement of the gaming platform. Under considerable pressure from psychologists and activists, this project, according to scientists and researchers, will allow the implementation of newly created approaches in such areas as education, medicine, sports, etc. without harming brain activity and with a significant improvement of communication skills in people of absolutely different ages in only 5 years.




\section{Nejaká časť} \label{nejaka}

Z obr.~\ref{f:rozhod} je všetko jasné. 

\begin{figure*}[tbh]
\centering
%\includegraphics[scale=1.0]{diagram.pdf}
Aj text môže byť prezentovaný ako obrázok. Stane sa z neho označný plávajúci objekt. Po vytvorení diagramu zrušte znak \texttt{\%} pred príkazom \verb|\includegraphics| označte tento riadok ako komentár (tiež pomocou znaku \texttt{\%}).
\caption{Rozhodujúci argument.}
\label{f:rozhod}
\end{figure*}



\section{Iná časť} \label{ina}

Základným problémom je teda\ldots{} Najprv sa pozrieme na nejaké vysvetlenie (časť~\ref{ina:nejake}), a potom na ešte nejaké (časť~\ref{ina:nejake}).\footnote{Niekedy môžete potrebovať aj poznámku pod čiarou.}

Môže sa zdať, že problém vlastne nejestvuje\cite{Coplien:MPD}, ale bolo dokázané, že to tak nie je~\cite{Czarnecki:Staged, Czarnecki:Progress}. Napriek tomu, aj dnes na webe narazíme na všelijaké pochybné názory\cite{PLP-Framework}. Dôležité veci možno \emph{zdôrazniť kurzívou}.


\subsection{Nejaké vysvetlenie} \label{ina:nejake}

Niekedy treba uviesť zoznam:

\begin{itemize}
\item jedna vec
\item druhá vec
	\begin{itemize}
	\item x
	\item y
	\end{itemize}
\end{itemize}

Ten istý zoznam, len číslovaný:

\begin{enumerate}
\item jedna vec
\item druhá vec
	\begin{enumerate}
	\item x
	\item y
	\end{enumerate}
\end{enumerate}


\subsection{Ešte nejaké vysvetlenie} \label{ina:este}

\paragraph{Veľmi dôležitá poznámka.}
Niekedy je potrebné nadpisom označiť odsek. Text pokračuje hneď za nadpisom.



\section{Dôležitá časť} \label{dolezita}




\section{Ešte dôležitejšia časť} \label{dolezitejsia}




\section{Záver} \label{zaver} % prípadne iný variant názvu



%\acknowledgement{Ak niekomu chcete poďakovať\ldots}


% týmto sa generuje zoznam literatúry z obsahu súboru literatura.bib podľa toho, na čo sa v článku odkazujete
\bibliography{literatura}
\bibliographystyle{plain} % prípadne alpha, abbrv alebo hociktorý iný
\end{document}
