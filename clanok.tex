% Metódy inžinierskej práce

\documentclass[10pt,twoside,slovak,a4paper]{article}

\usepackage[slovak]{babel}
%\usepackage[T1]{fontenc}
\usepackage[IL2]{fontenc} % lepšia sadzba písmena Ľ než v T1
\usepackage[utf8]{inputenc}
\usepackage{graphicx}
\usepackage{url} % príkaz \url na formátovanie URL
\usepackage{hyperref} % odkazy v texte budú aktívne (pri niektorých triedach dokumentov spôsobuje posun textu)

\usepackage{cite}
%\usepackage{times}

\pagestyle{headings}

\title{Multiplayer games: the opportunity to develop your social skills?\thanks{Semester project in the subject Methods of engineering work, the academic year 2022/23, leading: Vladimír Mlynarovič}} 

\author{Maryna Kolesnykova\\[2pt]
	{\small Slovenská technická univerzita v Bratislave}\\
	{\small Faculty of informatics and information technologies}\\
	{\small \texttt{xkolesnykova@stuba.sk}}
	}

\date{\small 6. December 2022} 



\begin{document}

\maketitle

\begin{abstract}

The article explores how multiplayer games influence people’s behavior and abilities to communicate with each other. Recent research on human interaction wouldn’t be completely clear without examining the field of online entertainment (computer gaming in particular). A brief overview of the psychologist’s conclusion and the professional player’s interview will form the base for the analysis. Another crucial point of this work is the target audience The focus then shifts back to team-player game development, highlighting the last few years of explosive growth and its impact on modern society. The difference between face-to-face communication and text or voice massaging is discussed by comparing brain activity while completing the required game tasks and normal physical ball games.
\end{abstract}



\section{Introduction}

  With the rapid progress in the field of information technology and the popularization of new and diverse devices (especially those used in the field of entertainment), it is difficult not to note the close intertwining of virtual reality and the real world. Communication using social networks has become an integral part of our lives. The growing demand for online entertainment (especially games) has forced developers, programmers, and other people, closely related to this field, to take a decisive step forward and spread the means of communication in online games. The most difficult part of the plan remained the implementation of convenient means of communication, which would be practical to use during the execution of the tasks. While the rest of the world is slowly gaining access to the technologies, traditions, and interactions of digital society, corporate organizations are striving to establish themselves as the main owners of cyberspace. The desire of corporations to dominate the Internet is motivated by the very nature of digitalization, which reinforces a social process in which the production and distribution of information become the most important economic activity of society, where information technology begins to function as a key infrastructure for all industrial production and service provision, and in which information itself becomes a product, which is sold worldwide. Comparing the advantages and disadvantages of individual areas of cyberspace development helps to understand how these areas of human activity and artificial intelligence affect modern society. The spread and popularization of eSports require constant improvement of communication tools, affecting home gamers' entertainment needs. The improvement of small spheres of cyberspace is reflected in the entire Internet community, especially children and teenagers, who, according to research, are the most active users of the Internet space.

  In the era of active development of information technologies and the creation of one's own "virtual self", communication in virtual reality and the development of communication skills in this environment is problem number one in terms of importance and concern among the population of European countries.  Texting is the main way of communicating in online gaming structures when voice and video chats have just started to populate mainstream gaming platforms. The ability to non-verbally perceive information, focusing on controlling simulations and other types of games, creates the basis for the development and implementation by entrepreneurs of a platform that provides guaranteed income in the future and acts as a trigger for the improvement of the gaming platform. Under considerable pressure from psychologists and activists, this project, according to scientists and researchers, will allow the implementation of newly created approaches in such areas as education, medicine, sports, etc. without harming brain activity and with a significant improvement of communication skills in people of absolutely different ages in only 5 years.




\section{History and development} 
 Online games have filled our lives extremely quickly. They have affected all areas of human life and continue to change our reality. So where did it all start? To begin with, it is necessary to understand what exactly multiplayer games mean. Multiplayer games refer to video games that can be played simultaneously by more than one person in the same gaming environment, either locally on the same computer or over a network, usually using the internet.

 The history of online gaming dates back to the 1970s. the first multiplayer games were text-based, particularly MUD1, created in 1978. Initially, it was available only on the Internal network, but in 1980 it was connected to the ARPANET, which made it able to try the first offline version. In the following decade, commercial games began to appear; the first was Islands of Kesmai - a role-playing computer game released in 1984. Games with a greater emphasis on graphics also began to be created: in particular, the action LINKS for the MSX, released in 1986, the aviation simulator Air Warrior (1987) and Go for the Famicom modem (1987) [The most popular games and their release dates can be seen in the table below ]



\section{Target audience}
  The Internet is one of the important tools for both academic and professional activities, as well as for leisure time. The biggest area of entertainment that the internet has to offer is online gaming. So who exactly is this type of entertainment for? Multiplayer games are interesting to many, they have found their users among men and women of different age categories and nationalities, they are used by both amateurs and professional e-sportsmen as a means of combating stress. According to statistics, the majority of players are children and teenagers aged 10 to 18, but it is worth noting that there are also adults 24-40 years old who are active players. Another fascinating fact is that due to researchers, female gamers are a rarer phenomenon than male gamers, and the increase in their number and the reason for their appearance gives reason to believe that they were encouraged to play by their social environment (for example, male friends, classmates, relatives). Nevertheless, the active social life of children and adolescents in the online space is unacceptable and incomprehensible, which will be discussed further in the section "Impact on social skills".



\section{Impact on social skills} 

  Shyness, lack of socialization skills, inability to find acquaintances or friends based on interests in one's place of residence, or certain physical disabilities that prevent normal communication - all these can be factors that lead people to come to online games. Multiplayer games are a way to get to find yourself for many people, and of course, they have a certain influence on human habits and sometimes even change the personality of the player. Any games have their positive and negative consequences, and the considered form of entertainment is no exception, so what changes do multiplayer online games make to the life of an ordinary person?

Môže sa zdať, že problém vlastne nejestvuje\cite{Coplien:MPD}, ale bolo dokázané, že to tak nie je~\cite{Czarnecki:Staged, Czarnecki:Progress}. Napriek tomu, aj dnes na webe narazíme na všelijaké pochybné názory\cite{PLP-Framework}. Dôležité veci možno \emph{zdôrazniť kurzívou}.


\subsection{Advantages} \label{ina:nejake}
  
  Studies have shown that playing video games every day can improve our visual attention, improve our task-switching efficiency, and make us faster at visual search and object discrimination. It also benefits many activities. Among the advantages of video games for children, spending time online is used in the context of education. They make learning interesting for children. Since multiplayer games are played as a team, any decision made, right or wrong, will have consequences for the members of the group, so video games develop empathy for other people. In a way, they promote healthy competition and increase self-esteem. When a player wins a game or advances to a new level, he involuntarily feels the happiness of the progress made. The game also helps improve our teamwork skills, as sometimes during the process, participants have to team up to overcome challenges. Since video games put characters in a situation where they have to use strategies to solve a problem, it can be concluded that the process encourages quick thinking and also guides other people to find a goal. The vast majority of online players speak English or another language that is often not native for most active users, so the player must learn to communicate with other people in the game, which encourages fans to learn new words and phrases.


\subsection{Disadvantages} \label{ina:este}

  Despite the many advantages of multiplayer games, it should be noted that they have no less negative consequences for their audience.It is very common to hear that a person becomes addicted to video games based on the amount of time they spend sitting, playing, and not paying attention to the world around them, but the most important part of getting rid of video game addiction remains to find the reason why they spend countless hours on their laptop or computer. In addition to Internet addiction, among the disadvantages of online games, an increase in the level of aggression is also highlighted. There is no scientific evidence that addicted teens or children always show signs of unstable mental health, but it is still important for parents to be informed about the games their child is playing. If a child spends hours playing video games that have violent content, this can cause negative consequences at the level of socialization. Spending a large number of hours in games where children, in particular, do not have direct contact with children of their own age. This can cause a certain negative drive in young people or children, causing players to no longer be able to distinguish between simulation and reality, leading to behavioral problems. Video games can make gamers impatient because things have to happen fast in virtual reality, so as a result,  gamers have trouble adjusting to real life. Another disadvantage is that young people or children spend so many hours in front of the screen that it can cause problems with falling asleep as well as nightmares.

\paragraph{Veľmi dôležitá poznámka.}
Niekedy je potrebné nadpisom označiť odsek. Text pokračuje hneď za nadpisom.



\section{Dôležitá časť} \label{dolezita}




\section{Ešte dôležitejšia časť} \label{dolezitejsia}




\section{Záver} \label{zaver} % prípadne iný variant názvu



%\acknowledgement{Ak niekomu chcete poďakovať\ldots}


% týmto sa generuje zoznam literatúry z obsahu súboru literatura.bib podľa toho, na čo sa v článku odkazujete
\bibliography{literatura}
\bibliographystyle{plain} % prípadne alpha, abbrv alebo hociktorý iný
\end{document}
